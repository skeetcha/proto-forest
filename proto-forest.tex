\documentclass{article}
\usepackage{fullpage}
\usepackage{tipa}
\usepackage{graphicx}
\usepackage{vowel}
\usepackage{makecell}
\usepackage{fontspec}
\setmainfont{Noto Sans}
\usepackage{environ}
\usepackage{xstring}
\usepackage[utf8]{inputenc}
\usepackage{array}
\usepackage{chngcntr}
\usepackage{fancyhdr}
\usepackage{multicol}

\newfontfamily\oghamfont{Noto Sans Ogham}
\NewEnviron{ogham}{%
    \oghamfont
    \StrSubstitute{\BODY}{b}{ᚁ}[\BODY]%
    \StrSubstitute{\BODY}{B}{ᚁ}[\BODY]%
    \StrSubstitute{\BODY}{l}{ᚂ}[\BODY]%
    \StrSubstitute{\BODY}{L}{ᚂ}[\BODY]%
    \StrSubstitute{\BODY}{f}{ᚃ}[\BODY]%
    \StrSubstitute{\BODY}{F}{ᚃ}[\BODY]%
    \StrSubstitute{\BODY}{s}{ᚄ}[\BODY]%
    \StrSubstitute{\BODY}{S}{ᚄ}[\BODY]%
    \StrSubstitute{\BODY}{n}{ᚅ}[\BODY]%
    \StrSubstitute{\BODY}{N}{ᚅ}[\BODY]%
    \StrSubstitute{\BODY}{h}{ᚆ}[\BODY]%
    \StrSubstitute{\BODY}{H}{ᚆ}[\BODY]%
    \StrSubstitute{\BODY}{d}{ᚇ}[\BODY]%
    \StrSubstitute{\BODY}{D}{ᚇ}[\BODY]%
    \StrSubstitute{\BODY}{t}{ᚈ}[\BODY]%
    \StrSubstitute{\BODY}{T}{ᚈ}[\BODY]%
    \StrSubstitute{\BODY}{c}{ᚉ}[\BODY]%
    \StrSubstitute{\BODY}{C}{ᚉ}[\BODY]%
    \StrSubstitute{\BODY}{m}{ᚋ}[\BODY]%
    \StrSubstitute{\BODY}{M}{ᚋ}[\BODY]%
    \StrSubstitute{\BODY}{g}{ᚌ}[\BODY]%
    \StrSubstitute{\BODY}{G}{ᚌ}[\BODY]%
    \StrSubstitute{\BODY}{r}{ᚏ}[\BODY]%
    \StrSubstitute{\BODY}{R}{ᚏ}[\BODY]%
    \StrSubstitute{\BODY}{p}{ᚊ}[\BODY]%
    \StrSubstitute{\BODY}{P}{ᚊ}[\BODY]%
    \StrSubstitute{\BODY}{a}{ᚐ}[\BODY]%
    \StrSubstitute{\BODY}{A}{ᚐ}[\BODY]%
    \StrSubstitute{\BODY}{i}{ᚔ}[\BODY]%
    \StrSubstitute{\BODY}{I}{ᚔ}[\BODY]%
    \StrSubstitute{\BODY}{o}{ᚑ}[\BODY]%
    \StrSubstitute{\BODY}{O}{ᚑ}[\BODY]%
    \StrSubstitute{\BODY}{u}{ᚒ}[\BODY]%
    \StrSubstitute{\BODY}{U}{ᚒ}[\BODY]%
    \StrSubstitute{\BODY}{e}{ᚓ}[\BODY]%
    \StrSubstitute{\BODY}{E}{ᚓ}[\BODY]%
    \StrSubstitute{\BODY}{á}{ᚕ}[\BODY]%
    \StrSubstitute{\BODY}{Á}{ᚕ}[\BODY]%
    \StrSubstitute{\BODY}{é}{ᚘ}[\BODY]%
    \StrSubstitute{\BODY}{É}{ᚘ}[\BODY]%
    \StrSubstitute{\BODY}{í}{ᚙ}[\BODY]%
    \StrSubstitute{\BODY}{Í}{ᚙ}[\BODY]%
    \StrSubstitute{\BODY}{ó}{ᚖ}[\BODY]%
    \StrSubstitute{\BODY}{Ó}{ᚖ}[\BODY]%
    \StrSubstitute{\BODY}{ú}{ᚗ}[\BODY]%
    \StrSubstitute{\BODY}{Ú}{ᚗ}[\BODY]%
    \StrSubstitute{\BODY}{ }{ }[\BODY]%
    \BODY
}

\title{Proto-Forest}
\author{Cass Forest}
\date{May 2024}

\counterwithin*{section}{part}

\newcommand{\entry}[2]{\markboth{#1}{#1}\item[#1] \begin{ogham} #1 \end{ogham} { }/\textipa{#2}/}
\newcommand{\definition}[2]{{ }\textit{#1} #2}

\begin{document}
\pagenumbering{gobble}
\maketitle
\newpage
\pagenumbering{roman}
\tableofcontents
\newpage
\pagenumbering{arabic}
\part{Introduction}
Introduction, history, etc.
\newpage
\part{Phonology}
\section{Consonants}
\begin{center}
\scalebox{0.7}{
\begin{tabular}{l|c|c|c|c|c|c|c|c}
& \textbf{Bilabial} & \textbf{Labiodental} & \textbf{Dental} & \textbf{Alveolar} & \textbf{Post-alveolar} & \textbf{Palatal} & \textbf{Velar} & \textbf{Glottal} \\ \hline
\textbf{Nasal} & \textipa{m\super{j}} \textipa{m\super G} & & \textipa{\textsubbridge{n}\super G} & \textipa{n\super{j}} & & \textipa{\textltailn} & \textipa{N} & \\
\textbf{Plosive} & \textipa{p\super{j}} \textipa{p\super G} \textipa{b\super{j}} \textipa{b\super G} & & \textipa{\textsubbridge{t}\super G} \textipa{\textsubbridge{d}\super G} & \textipa{t\super{j}} \textipa{d\super{j}} & & \textipa{c} \textipa{\textbardotlessj} & \textipa{k} \textipa{g} & \\
\textbf{Fricative} & & \textipa{f\super{j}} \textipa{f\super G} \textipa{v\super{j}} \textipa{v\super G} & & \textipa{s\super G} & \textipa{S} & \textipa{\c{c}} & \textipa{x} & \textipa{h} \\
\makecell[l]{\textbf{Lateral} \\ \textbf{approximant}} & & & & \textipa{l\super{j}} \textipa{l\super G} & & & & \\
\textbf{Flap} & & & & \textipa{R\super{j}} \textipa{R\super G} & & & & \\
\end{tabular}
}
\end{center}
\section{Vowels}
\subsection{Monophthongs}
\begin{center}
\begin{vowel}
\putcvowel{\textipa{i:}}{1}
\putcvowel{\textipa{I}}{13}
\putcvowel{\textipa{e:}}{2}
\putcvowel{\textipa{E}}{3}
\putcvowel{\textipa{a}}{4}
\putcvowel{\textipa{U}}{14}
\putcvowel{\textipa{u:}}{8}
\putcvowel{\textipa{o:}}{7}
\putcvowel{\textipa{O}}{6}
\putcvowel{\textipa{A:}}{5}
\end{vowel}
\end{center}
\subsection{Diphthongs}
\textipa{u\textsubarch{@}} \textipa{@\textsubarch{i}} \textipa{i\textsubarch{@}} \textipa{@\textsubarch{u}}
\section{Stress}
Stress falls on the first syllable.
\section{Phonotactics}
\subsection{Word-initial consonant clusters}
Words can begin with clusters of two or three consonants. In general, all the consonants in a cluster agree in their quality, i.e. either all are broad or all are slender. Two-consonant clusters consist of an obstruent consonant followed by a liquid or nasal consonant (however, labial consonants may not be followed by a nasal). In addition, /\textipa{s\super G}/ and /\textipa{S}/ may be followed by a voiceless stop. Three-consonant clusters consist of /\textipa{s\super G}/ or /\textipa{S}/ plus a voiceless stop plus a liquid.
\subsection{Post-vocalic consonant clusters and epenthesis}
Like word-initial consonant clusters, post-vocalic consonant clusters usually agree in broad or slender quality.

A cluster of /\textipa{R\super G}, \textipa{R\super{j}}/, /\textipa{l\super G}, \textipa{l\super{j}}/, or /\textipa{\textsubbridge{n}\super G}, \textipa{n\super{j}}/ followed by a labial or dorsal consonant (except the voiceless stops /\textipa{p\super G}, \textipa{p\super{j}}/, /\textipa{k}, \textipa{c}/) is broken up by an epenthetic vowel /\textipa{@}/. There is no epenthesis, however, if the vowel preceding the cluster is long or a diphthong. There is also no epenthesis into words that are at least three syllables long.
\newpage
\part{Morphology}
The morphology of Proto-Forest depends on what kind of vowel or consonant the word starts with. "SC" in the tables below refers to the word starting with a slender consonant. "BC" in the tables below refers to the word starting with a broad consonant. "SV" in the tables below refers to the word starting with a slender vowel. "BV" in the tables below refers to the word starting with a broad vowel.
\section{Nouns}
Nouns are classed into two genders: Masculine and Feminine. Masculine nouns are nouns that end in a broad consonant or vowel. Feminine nouns are nouns that end in a slender consonant or vowel. "Redup" in the tables below means to reduplicate the first syllable onto the start of the word.
\subsection{Masculine Noun Declension}
\begin{center}
\begin{tabular}{c|m{2in}|m{2in}}
& \textbf{Singular} & \textbf{Plural} \\ \hline
\textbf{Nominative} & No affix & Redup \\ \hline
\textbf{Accusative} & \makecell[l]{Prefix /\textipa{e:l\super{j}}/ (SV) \\ Prefix /\textipa{e:@\textsubarch{i}l\super G} (BV) \\ Prefix /\textipa{e:l\super{j}E}/ (SC) \\ Prefix /\textipa{e:l\super{j}a}/ (BC)} & \makecell[l]{Redup + Prefix /\textipa{e:l\super{j}}/ (SV) \\ Redup + Prefix /\textipa{e:@\textsubarch{i}l\super G} (BV) \\ Redup + Prefix /\textipa{e:l\super{j}E}/ (SC) \\ Redup + Prefix /\textipa{e:l\super{j}a}/ (BC)} \\ \hline
\textbf{Genitive} & \makecell[l]{Prefix /\textipa{SaIn\super{j}}/ (SV) \\ Prefix /\textipa{Sa\textsubbridge{n}\super G}/ (BV) \\ Prefix /\textipa{SaI}/ (SC) \\ Prefix /\textipa{Sa}/ (BC)} & \makecell[l]{Redup + Prefix /\textipa{SaIn\super{j}}/ (SV) \\ Redup + Prefix /\textipa{Sa\textsubbridge{n}\super G}/ (BV) \\ Redup + Prefix /\textipa{SaI}/ (SC) \\ Redup + Prefix /\textipa{Sa}/ (BC)} \\
\end{tabular}
\end{center}
\subsubsection{Example}
\begin{center}
\begin{ogham}
ífígrian\\
\end{ogham}
ífígrian\\
/\textipa{i:f\super{j}i:{\textbardotlessj}R\super{j}i\textsubarch{@}\textsubbridge{n}\super G}/

\begin{tabular}{c|c|c}
& \textbf{Singular} & \textbf{Plural} \\ \hline
\textbf{Nominative} & \makecell{\begin{ogham} ífígrian \end{ogham} \\ ífígrian \\ /\textipa{i:f\super{j}i:{\textbardotlessj}R\super{j}i\textsubarch{@}\textsubbridge{n}\super G}/} & \makecell{\begin{ogham} ífífígrian \end{ogham} \\ ífífígrian \\ /\textipa{i:f\super{j}i:f\super{j}i:{\textbardotlessj}R\super{j}i\textsubarch{@}\textsubbridge{n}\super G}/} \\ \hline
\textbf{Accusative} & \makecell{\begin{ogham} élífígrian \end{ogham} \\ élífígrian \\ /\textipa{e:l\super{j}i:f\super{j}i:{\textbardotlessj}R\super{j}i\textsubarch{@}\textsubbridge{n}\super G}/} & \makecell{\begin{ogham} élífífígrian \end{ogham} \\ élífífígrian \\ /\textipa{e:l\super{j}i:f\super{j}i:f\super{j}i:{\textbardotlessj}R\super{j}i\textsubarch{@}\textsubbridge{n}\super G}/} \\ \hline
\textbf{Genitive} & \makecell{\begin{ogham} seainífígrian \end{ogham} \\ seainífígrian \\ /\textipa{SaIn\super{j}i:f\super{j}i:{\textbardotlessj}R\super{j}i\textsubarch{@}\textsubbridge{n}\super G}/} & \makecell{\begin{ogham} seanífífígrian \end{ogham} \\ seanífífígrian \\ /\textipa{SaIn\super{j}i:f\super{j}i:f\super{j}i:{\textbardotlessj}R\super{j}i\textsubarch{@}\textsubbridge{n}\super G}/} \\
\end{tabular}
\end{center}
\subsection{Feminine Noun Declension}
\begin{center}
\begin{tabular}{c|p{2in}|p{2in}}
& \textbf{Singular} & \textbf{Plural} \\ \hline
\textbf{Nominative} & No affix & \makecell[l]{Prefix /\textipa{R\super{j}}/ (SV) \\ Prefix /\textipa{R\super G}/ (BV) \\ Prefix /\textipa{R\super{j}i:}/ (SC) \\ Prefix /\textipa{R\super{j}i:@\textsubarch{i}}/ (BC)} \\ \hline
\textbf{Accusative} & \makecell[l]{Suffix /\textipa{n\super{j}i:}/ (SV) \\ Suffix /\textipa{i:}/ (BV/SC) \\ Suffix /\textipa{@\textsubarch{i}i:}/ (BC)} & \makecell[l]{Prefix /\textipa{R\super{j}}/ + Suffix /\textipa{n\super{j}i:}/ (SV) \\ Prefix /\textipa{R\super G}/ + Suffix /\textipa{i:}/ (BV) \\ Prefix /\textipa{R\super{j}i:}/ + Suffix /\textipa{i:}/ (SC) \\ Prefix /\textipa{R\super{j}i:@\textsubarch{i}}/ + Suffix /\textipa{@\textsubarch{i}i:}/ (BC)} \\ \hline
\textbf{Genitive} & \makecell[l]{Suffix /\textipa{@\textsubarch{i}}/ (BV/SV/BC) \\ Suffix /\textipa{i\textsubarch{@}}/ (SC)} & \makecell[l]{Prefix /\textipa{R\super{j}}/ + Suffix /\textipa{@\textsubarch{i}}/ (SV) \\ Prefix /\textipa{R\super G}/ + Suffix /\textipa{@\textsubarch{i}}/ (BV) \\ Prefix /\textipa{R\super{j}i:}/ + Suffix /\textipa{i\textsubarch{@}}/ (SC) \\ Prefix /\textipa{R\super{j}i:@\textsubarch{i}}/ + Suffix /\textipa{@\textsubarch{i}}/ (BC)}
\end{tabular}
\end{center}
\subsubsection{Example}
\begin{center}
\begin{ogham}
gí\\
\end{ogham}
gí\\
/\textipa{{\textbardotlessj}i:}/

\begin{tabular}{c|c|c}
& \textbf{Singular} & \textbf{Plural} \\ \hline
\textbf{Nominative} & \makecell{\begin{ogham} gí \end{ogham} \\ gí \\ /\textipa{{\textbardotlessj}i:}/} & \makecell{\begin{ogham} rígí \end{ogham} \\ rígí \\ /\textipa{R\super{j}i:{\textbardotlessj}i:}/} \\ \hline
\textbf{Accusative} & \makecell{\begin{ogham} gíní \end{ogham} \\ gíní \\ /\textipa{{\textbardotlessj}i:n\super{j}i:}/} & \makecell{\begin{ogham} rígíní \end{ogham} \\ rígíní \\ /\textipa{R\super{j}i:{\textbardotlessj}i:n\super{j}i:}/} \\ \hline
\textbf{Genitive} & \makecell{\begin{ogham} gía \end{ogham} \\ gía \\ /\textipa{{\textbardotlessj}i:@\textsubarch{i}}/} & \makecell{\begin{ogham} rígía \end{ogham} \\ rígía \\ /\textipa{R\super{j}i:{\textbardotlessj}i:@\textsubarch{i}}/} \\
\end{tabular}
\end{center}
\section{Pronouns}
All pronouns end in vowels.

\begin{center}
\begin{tabular}{c|c|c}
\textbf{Nominative} & \textbf{Accusative} & \textbf{Genitive} \\ \hline
\makecell{Suffix /\textipa{n\super{j}}/ (SV) \\ Suffix /\textipa{\textsubbridge{n}\super G}/ (BV)} & \makecell{Suffix /\textipa{In\super{j}}/ (BV) \\ Suffix /\textipa{@\textsubarch{i}\textsubbridge{n}\super G}/ (SV)} & \makecell{Suffix /\textipa{\textsubbridge{n}\super G}/ (BV) \\ Suffix /\textipa{@\textsubarch{i}\textsubbridge{n}\super{G}@\textsubarch{i}}/ (SV)} \\
\end{tabular}
\end{center}
\subsection{Example}
\begin{center}
\begin{ogham}
ei\\
\end{ogham}
ei\\
/\textipa{E}/

\begin{tabular}{c|c|c}
\textbf{Nominative} & \textbf{Accusative} & \textbf{Genitive} \\ \hline
\makecell{\begin{ogham} ein \end{ogham} \\ ein \\ /\textipa{En\super{j}}/} & \makecell{\begin{ogham} eian \end{ogham} \\ eian \\ /\textipa{E@\textsubarch{i}\textsubbridge{n}\super G}/} & \makecell{\begin{ogham} eiana \end{ogham} \\ eiana \\ /\textipa{E@\textsubarch{i}\textsubbridge{n}\super{G}@\textsubarch{i}}/} \\
\end{tabular}
\end{center}
\section{Verbs}
\begin{center}
\begin{tabular}{c|c|c|c}
& \textbf{PRES} & \textbf{PAST} & \textbf{FUT} \\ \hline
\textbf{1.SG} & \makecell{Prefix /\textipa{f\super{G}u:}/ (BC/SV/BV) \\ Prefix /\textipa{f\super{G}u:I}/ (SC)} & \makecell{Prefix /\textipa{t\super{j}}/ (SV) \\ Prefix /\textipa{\textsubbridge{t}\super{G}}/ (BV) \\ Prefix /\textipa{t\super{j}i:}/ (SC) \\ Prefix /\textipa{t\super{j}i:@\textsubarch{i}}/ (BC)} & \makecell{Prefix /\textipa{m\super{G}}/ (BV) \\ Prefix /\textipa{m\super{j}}/ (SV) \\ Prefix /\textipa{m\super{G}@\textsubarch{i}}/ (BC) \\ Prefix /\textipa{m\super{G}@\textsubarch{u}}/ (SC)} \\ \hline
\textbf{2.SG} & \makecell{Prefix /\textipa{ce:}/ (SC/BV/SV) \\ Prefix /\textipa{ce:@\textsubarch{i}}/ (BC)} & \makecell{Prefix /\textipa{k}/ (BV) \\ Prefix /\textipa{c}/ (SV) \\ Prefix /\textipa{ko:}/ (BC) \\ Prefix /\textipa{ko:I}/ (SC)} & \makecell{Prefix /\textipa{s\super{G}k}/ (BV) \\ Prefix /\textipa{Sc}/ (SV) \\ Prefix /\textipa{s\super{G}kA:}/ (BC) \\ Prefix /\textipa{s\super{G}kA:I}/ (SC)} \\ \hline
\textbf{3.SG} & \makecell{Prefix /\textipa{k}/ (BV) \\ Prefix /\textipa{c}/ (SV) \\ Prefix /\textipa{ca}/ (BC) \\ Prefix /\textipa{caI}/ (SC)} & \makecell{Prefix /\textipa{d\super{j}R\super{j}E}/ (SV/BV/SC) \\ Prefix /\textipa{d\super{j}R\super{j}E@\textsubarch{i}}/ (BC)} & \makecell{Prefix /\textipa{U\textsubbridge{n}\super{G}}/ (BV) \\ Prefix /\textipa{In\super{j}}/ (SV) \\ Prefix /\textipa{U\textsubbridge{n}\super{G}@\textsubarch{i}}/ (BC) \\ Prefix /\textipa{U\textsubbridge{n}\super{G}@\textsubarch{u}}/ (SC)} \\ \hline
\textbf{1.PL} & \makecell{Prefix /\textipa{b\super{G}}/ (BV) \\ Prefix /\textipa{b\super{j}}/ (SV) \\ Prefix /\textipa{b\super{j}E}/ (SC) \\ Prefix /\textipa{b\super{j}a}/ (BC)} & \makecell{Prefix /\textipa{k}/ (BV) \\ Prefix /\textipa{c}/ (SV) \\ Prefix /\textipa{ku\textsubarch{@}}/ (BC) \\ Prefix /\textipa{kI}/ (SC)} & \makecell{Prefix /\textipa{i:\c{c}}/ (SV) \\ Prefix /\textipa{i:@\textsubarch{i}x}/ (BV) \\ Prefix /\textipa{i:\c{c}E}/ (SC) \\ Prefix /\textipa{i:\c{c}a}/ (BC)} \\ \hline
\textbf{2.PL} & \makecell{Prefix /\textipa{x}/ (BV) \\ Prefix /\textipa{\c{c}}/ (SV) \\ Prefix /\textipa{xo:}/ (BC) \\ Prefix /\textipa{xo:I}/ (SC)} & \makecell{Prefix /\textipa{kR\super{G}}/ (BV) \\ Prefix /\textipa{cR\super{j}}/ (SV) \\ Prefix /\textipa{cR\super{j}I}/ (SC) \\ Prefix /\textipa{cR\super{j}i\textsubarch{@}}/ (BC)} & \makecell{Prefix /\textipa{g}/ (BV) \\ Prefix /\textipa{\textbardotlessj}/ (SV) \\ Prefix /\textipa{gu\textsubarch{@}}/ (BC) \\ Prefix /\textipa{gI}/ (SC)} \\ \hline
\textbf{3.PL} & \makecell{Prefix /\textipa{\textsubbridge{d}\super{G}}/ (BV) \\ Prefix /\textipa{d\super{j}}/ (SV) \\ Prefix /\textipa{d\super{j}i\textsubarch{@}}/ (BC) \\ Prefix /\textipa{d\super{j}I}/ (SC)} & \makecell{Prefix /\textipa{@\textsubarch{i}x\textsubbridge{t}\super{G}}/ (BV) \\ Prefix /\textipa{@\textsubarch{u}\c{c}t\super{j}}/ (SV) \\ Prefix /\textipa{@\textsubarch{i}x\textsubbridge{t}\super{G}@\textsubarch{i}}/ (BC) \\ Prefix /\textipa{@\textsubarch{u}\c{c}t\super{j}I}/ (SC)} & \makecell{Prefix /\textipa{A:m\super{G}}/ (BV) \\ Prefix /\textipa{A:Im\super{j}}/ (SV) \\ Prefix /\textipa{A:Im\super{j}i:}/ (SC) \\ Prefix /\textipa{A:m\super{G}@\textsubarch{i}}/ (BC)} \\
\end{tabular}
\end{center}
\subsection{Example}
\begin{center}
\begin{ogham}
chú\\
\end{ogham}
chú\\
/\textipa{xu:}/

\begin{tabular}{c|c|c|c}
& \textbf{PRES} & \textbf{PAST} & \textbf{FUT} \\ \hline
\textbf{1.SG} & \makecell{\begin{ogham} fúchú \end{ogham} \\ fúchú \\ /\textipa{f\super{G}u:xu:}/} & \makecell{\begin{ogham} tíachú \end{ogham} \\ tíachú \\ /\textipa{t\super{j}i:@\textsubarch{i}xu:}/} & \makecell{\begin{ogham} machú \end{ogham} \\ machú \\ /\textipa{m\super{G}@\textsubarch{i}xu:}/} \\ \hline
\textbf{2.SG} & \makecell{\begin{ogham} céachú \end{ogham} \\ céachú \\ /\textipa{ce:@\textsubarch{i}xu:}/} & \makecell{\begin{ogham} cóchú \end{ogham} \\ cóchú \\ /\textipa{ko:xu:}/} & \makecell{\begin{ogham} scáchú \end{ogham} \\ scáchú \\ /\textipa{s\super{G}kA:xu:}/} \\ \hline
\textbf{3.SG} & \makecell{\begin{ogham} ceachú \end{ogham} \\ ceachú \\ /\textipa{caxu:}/} & \makecell{\begin{ogham} dreiachú \end{ogham} \\ dreiachú \\ /\textipa{d\super{j}R\super{j}E@\textsubarch{i}xu:}/} & \makecell{\begin{ogham} unachú \end{ogham} \\ unachú \\ /\textipa{U\textsubbridge{n}\super{G}xu:}/} \\ \hline
\textbf{1.PL} & \makecell{\begin{ogham} beachú \end{ogham} \\ beachú \\ /\textipa{b\super{j}axu:}/} & \makecell{\begin{ogham} cuachú \end{ogham} \\ cuachú \\ /\textipa{ku\textsubarch{@}xu:}/} & \makecell{\begin{ogham} ícheachú \end{ogham} \\ ícheachú \\ /\textipa{i:\c{c}axu:}/} \\ \hline
\textbf{2.PL} & \makecell{\begin{ogham} chóchú \end{ogham} \\ chóchú \\ /\textipa{xo:xu:}/} & \makecell{\begin{ogham} criachú \end{ogham} \\ criachú \\ /\textipa{cR\super{j}i\textsubarch{@}xu:}/} & \makecell{\begin{ogham} guachú \end{ogham} \\ guachú \\ /\textipa{gu\textsubarch{@}xu:}/} \\ \hline
\textbf{3.PL} & \makecell{\begin{ogham} diachú \end{ogham} \\ diachú \\ /\textipa{d\super{j}i\textsubarch{@}xu:}/} & \makecell{\begin{ogham} achtachú \end{ogham} \\ achtachú \\ /\textipa{@\textsubarch{i}x\textsubbridge{t}\super{G}@\textsubarch{i}xu:}/} & \makecell{\begin{ogham} ámachú \end{ogham} \\ ámachú \\ /\textipa{A:m\super{G}@\textsubarch{i}xu:}/} \\
\end{tabular}
\end{center}
\newpage
\part{Derivational Morphology}
To turn an adjective into an adverb, suffix /\textipa{@\textsubarch{i}}/ if it ends in a broad consonant or vowel, or /\textipa{a}/ if it ends in a slender consonant or vowel.

To turn an adjective into a noun denoting the quality of being that adjective, suffix /\textipa{R\super{G}} if it ends in a broad vowel, /\textipa{R\super{j}}/ if it ends in a slender vowel, /\textipa{@\textsubarch{i}R\super{G}}/ if it ends in a broad consonant, or /\textipa{aR\super{G}}/ if it ends in a slender consonant.

To turn an adjective into a verb denoting the action of making something that adjective, suffix /\textipa{\textltailn}/ if it ends in a slender vowel, /\textipa{N}/ if it ends in a broad vowel, /\textipa{@\textsubarch{i}N}/ if it ends in a broad consonant, or /\textipa{aN}/ if it ends in a slender consonant.

To turn a noun into an adjective denoting something having the quality of that noun, suffix /\textipa{\textsubbridge{d}\super{G}} if it ends in a broad vowel, /\textipa{d\super{j}}/ if it ends in a slender vowel, /\textipa{u\textsubarch{@}\textsubbridge{d}\super{G}}/ if it ends in a broad consonant, or /\textipa{i\textsubarch{@}\textsubbridge{d}\super{G}}/ if it ends in a slender consonant.

To turn a noun into an adjective relating to that noun (e.g. economy to economic), suffix /\textipa{i\textsubarch{@}}/ if it ends in a slender consonant or any vowel, or /\textipa{@\textsubarch{i}}/ if it ends in a broad consonant.

To turn a noun into a verb, suffix /\textipa{g}/ if it ends in a broad vowel, /\textipa{\textbardotlessj}/ if it ends in a slender vowel, /\textipa{@\textsubarch{i}g}/ if it ends in a broad consonant, or /\textipa{i\textsubarch{@}g}/ if it ends in a slender consonant.

To turn a verb into an adjective denoting the result of doing that verb, prefix /\textipa{\textsubbridge{t}\super{G}o:} if it starts with a broad consonant or any vowel, or /\textipa{\textsubbridge{t}o:I}/ if it starts with a slender consonant.

To turn a verb into another verb denoting the tendenancy to do that verb, prefix /\textipa{aR\super{G}s\super{G}}/ if it starts with a broad vowel, /\textipa{@\textsubarch{u}R\super{j}S}/ if it starts with a slender vowel, /\textipa{@\textsubarch{u}R\super{j}Se:}/ if it starts with a slender consonant, or /\textipa{@\textsubarch{u}R\super{j}Se:@\textsubarch{i}}/ if it starts with a broad consonant.

To turn a verb into a noun denoting the act of doing the verb, prefix /\textipa{i:f\super{j}}/ if it starts with a slender vowel, /\textipa{i:@\textsubarch{i}f\super{G}}/ if it starts with a broad vowel, /\textipa{i:f\super{j}i:}/ if it starts with a slender consonant, or /\textipa{i:f\super{j}i:@\textsubarch{i}}/ if it starts with a broad consonant.

To turn a verb into a noun that that verb produces (e.g. know to knowledge), prefix /\textipa{A:}/ if it starts with a broad consonant or any vowel, or /\textipa{A:I}/ if it starts with a slender consonant.

To turn a verb into a noun denoting one who does that verb (e.g. paint to painter), suffix /\textipa{@\textsubarch{i}}/ if it ends in a broad consonant or vowel, or /\textipa{a}/ if it ends in a slender consonant or vowel.

To turn a noun into a noun that denotes the place where that noun takes place (e.g. wine to winery), suffix /\textipa{R\super{j}t\super{j}}/ if it ends with a slender vowel, /\textipa{R\super{G}\textsubbridge{t}\super{G}}/ if it ends with a broad vowel, /\textipa{i\textsubarch{@}R\super{G}\textsubbridge{t}\super{G}}/ if it ends with a slender consonant, or /\textipa{@\textsubarch{u}R\super{j}t\super{j}} if it ends with a broad consonant.

To turn a word into its dimunitive form, suffix /\textipa{St\super{j}}/ if it ends in a slender vowel, /\textipa{s\super{G}\textsubbridge{t}\super{G}}/ if it ends in a broad vowel, /\textipa{e:St\super{j}}/ if it ends in a slender consonant, or /\textipa{e:@\textsubarch{i}s\super{G}\textsubbridge{t}\super{G}} if it ends in a broad consonant.

To turn a word into its augmentative form, suffix /\textipa{I}/ if it ends in a slender consonant or any vowel, or /\textipa{@\textsubarch{u}}/ if it ends in a broad consonant.
\newpage
\part{Syntax}
Proto-Forest uses VSO word order and uses prepositions. Double negatives remain negative. Nouns precede both articles, determiners, and adjectives, but come after numbers and pronouns. When a noun possess's another noun, the possessee comes first. There is no indefinite article, but there is a definite article.
\newpage
\part{Orthography}
Broad consonants must appear next to broad vowels. Slender consonants must appear next to slender vowels. The following consonants are broad consonants and vowels.
\begin{center}
\begin{tabular}{c|c|c}
\textbf{Sound} & \textbf{Ogham} & \textbf{Romanization} \\ \hline
\textipa{b\super{G}} & \begin{ogham} b \end{ogham} & b \\
\textipa{p\super{G}} & \begin{ogham} p \end{ogham} & p \\
\textipa{m\super{G}} & \begin{ogham} m \end{ogham} & m \\
\textipa{f\super{G}} & \begin{ogham} f \end{ogham} & f \\ 
\textipa{v\super{G}} & \begin{ogham} bh \end{ogham} / \begin{ogham} mh \end{ogham} & bh / mh \\
\textipa{\textsubbridge{n}\super{G}} & \begin{ogham} n \end{ogham} & n \\
\textipa{\textsubbridge{t}\super{G}} & \begin{ogham} t \end{ogham} & t \\
\textipa{\textsubbridge{d}\super{G}} & \begin{ogham} d \end{ogham} & d \\
\textipa{s\super{G}} & \begin{ogham} s \end{ogham} & s \\
\textipa{l\super{G}} & \begin{ogham} l \end{ogham} & l \\
\textipa{R\super{G}} & \begin{ogham} r \end{ogham} & r \\
\textipa{N} & \begin{ogham} ng \end{ogham} & ng \\
\textipa{k} & \begin{ogham} c \end{ogham} & c \\
\textipa{g} & \begin{ogham} g \end{ogham} & g \\
\textipa{x} & \begin{ogham} ch \end{ogham} & ch \\
\textipa{@\textsubarch{i}} & \begin{ogham} a \end{ogham} & a \\
\textipa{O} & \begin{ogham} o \end{ogham} & o \\
\textipa{U} & \begin{ogham} u \end{ogham} & u \\
\textipa{A:} & \begin{ogham} á \end{ogham} & á \\
\textipa{o:} & \begin{ogham} ó \end{ogham} & ó \\
\textipa{u:} & \begin{ogham} ú \end{ogham} & ú \\
\end{tabular}
\end{center}

The following are slender consonants and vowels.

\begin{center}
\begin{tabular}{c|c|c}
\textbf{Sound} & \textbf{Ogham} & \textbf{Romanization} \\ \hline
\textipa{b\super{j}} & \begin{ogham} b \end{ogham} & b \\
\textipa{p\super{j}} & \begin{ogham} p \end{ogham} & p \\
\textipa{m\super{j}} & \begin{ogham} m \end{ogham} & m \\
\textipa{f\super{j}} & \begin{ogham} f \end{ogham} & f \\
\textipa{v\super{j}} & \begin{ogham} bh \end{ogham} / \begin{ogham} mh \end{ogham} & bh / mh \\
\textipa{n\super{j}} & \begin{ogham} n \end{ogham} & n \\
\textipa{t\super{j}} & \begin{ogham} t \end{ogham} & t \\
\textipa{d\super{j}} & \begin{ogham} d \end{ogham} & d \\
\textipa{S} & \begin{ogham} s \end{ogham} & s \\
\textipa{l\super{j}} & \begin{ogham} l \end{ogham} & l \\
\textipa{R\super{j}} & \begin{ogham} r \end{ogham} & r \\
\textipa{\textltailn} & \begin{ogham} ng \end{ogham} & ng \\
\textipa{c} & \begin{ogham} c \end{ogham} & c \\
\textipa{\textbardotlessj} & \begin{ogham} g \end{ogham} & g \\
\textipa{\c{c}} & \begin{ogham} ch \end{ogham} & ch \\
\textipa{I} & \begin{ogham} i \end{ogham} & i \\
\textipa{E} & \begin{ogham} e \end{ogham} & e \\
\textipa{i:} & \begin{ogham} í \end{ogham} & í \\
\textipa{e:} & \begin{ogham} é \end{ogham} & é \\
\end{tabular}
\end{center}

The consonant /\textipa{h}/ (\begin{ogham} h \end{ogham}) (h) is neither broad nor slender.

As for diphthongs...

\begin{center}
\begin{tabular}{c|c|c|c|c}
\textbf{Sound} & \textbf{Ogham} & \textbf{Romanization} & \textbf{Beginning Quality} & \textbf{Ending Quality} \\ \hline
\textipa{i\textsubarch{@}} & \begin{ogham} ia \end{ogham} & ia & Slender & Broad \\
\textipa{@\textsubarch{u}} & \begin{ogham} ai \end{ogham} & ai & Slender & Broad \\
\textipa{u\textsubarch{@}} & \begin{ogham} ua \end{ogham} & ua & Broad & Broad \\
\end{tabular}
\end{center}
\newpage
\part{Examples}
\newpage
\part{Lexicon}
\newpage
\setlength{\headsep}{30pt}
\setlength{\columnsep}{0.1in}
\begin{center}
\section*{A}
\end{center}
\pagestyle{fancy}
\fancyhead[L]{\textsf{\rightmark}} % Top left header
\fancyhead[R]{\textsf{\leftmark}} % Top right header
\renewcommand{\headrulewidth}{1.4pt} % Rule under the header
\fancyfoot[C]{\textbf{\textsf{\thepage}}} % Bottom center footer
\renewcommand{\footrulewidth}{1.4pt} % Rule under the footer
\footnotesize
\begin{multicols}{2}
\begin{description}
\entry{a}{@\textsubarch{i}} \definition{n.}{Life, a person's being. Also refers to a period of time from birth to death.} \definition{num.}{Fourteen.}
\entry{abh}{@\textsubarch{i}v\super{G}} \definition{num.}{Fifteen.}
\entry{ad}{@\textsubarch{i}\textsubbridge{d}\super{G}} \definition{v.}{Improve, to bring into a more desirable or excellent condition.}
\entry{áigilea}{A:I{\textbardotlessj}Il\super{j}a} \definition{n.}{Insurance, coverage by contract in which one party agrees to indemnify or reimburse another for loss that occurs under the terms of the contract.}
\entry{ainsear}{@\textsubarch{i}n\super{j}SaR\super{G}} \definition{n.}{Heat, a relatively high degree of warmth.}
\entry{aisea}{@\textsubarch{u}Sa} \definition{n.}{Sailor, a person whose occupation is sailing or navigation; mariner.}
\entry{aiseó}{@\textsubarch{u}So:} \definition{n.}{Sorrow, distress caused by loss, affliction, disappointment, etc.; grief, sadness, or regret.}
\entry{áitria}{A:It\super{j}R\super{j}i@\textsubarch{@}} \definition{Meeting, an assembly or conference of persons for a specific purpose.}
\entry{ámú}{A:m\super{G}u:} \definition{n.}{Certificate, a document serving as evidence or as written testimony, as of status, qualifications, privileges, or the truth of something.}
\entry{arsaogóc}{@\textsubarch{i}R\super{G}s\super{G}e:go:k} \definition{adj.}{having the property of adhering, as glue; adhesive.}
\entry{ást}{A:s\super{G}\textsubbridge{t}\super{G}} \definition{v.}{Compare, to examine (two or more objects, ideas, people, etc.) in order to note similarities and differences.}
\end{description}
\end{multicols}
\begin{center}
\section*{B}
\end{center}
\begin{multicols}{2}
\begin{description}
\entry{bacha}{b\super{G}@\textsubarch{i}x@\textsubarch{i}} \definition{n.}{Pepper, any plant of the genus Piper.}
\entry{bam}{b\super{G}@\textsubarch{i}m\super{G}} \definition{n.}{Brother, a male offspring having both parents in common with another offspring; a male sibling.} \definition{prep.}{Like, in like manner with; similarly to; in the manner characteristic of.}
\entry{bas}{b\super{G}@\textsubarch{i}s\super{G}} \definition{v.}{Get, to receive or come to have possession, use, or enjoyment of.}
\entry{béatha}{b\super{j}e:@\textsubarch{i}h@\textsubarch{i}} \definition{n.}{Bridge, a structure spanning and providing passage over a river, chasm, road, or the like.}
\entry{beibh}{b\super{j}Ev\super{j}} \definition{n.}{Gold, a precious yellow metallic element, highly malleable and ductile, and not subject to oxidation or corrosion.}
\entry{bheist}{v\super{j}ESt\super{j}} \definition{adj.}{Weird, strange; odd; bizarre.}
\entry{bhó}{v\super{G}o:} \definition{n.}{Boundary, a line or limit where one thing ends and another begins, or something that indicates such a line or limit. Also refers to Edge, a line or border at which a surface terminates; Border, the part or edge of a surface or area that forms its outer boundary; Mouth, the outfall at the lower end of a river or stream, where flowing water is discharged, as into a lake, sea, or ocean.}
\entry{bhoirei}{v\supper{G}OIR\super{j}E} \definition{n.}{Fashion, a prevailing custom or style of dress, etiquette, socializing, etc.}
\entry{bi}{b\super{j}I} \definition{v.}{Involve, to include as a necessary circumstance, condition, or consequence; imply; entail.}
\entry{biath}{b\super{j}i\textsubarch{@}h} \definition{n.}{Brush, an implement consisting of bristles, hair, or the like, set in or attached to a handle, used for painting, cleaning, polishing, grooming, etc.}
\entry{bó}{b\super{G}o:} \definition{n.}{Image, a physical likeness or representation of a person, animal, or thing, photographed, painted, sculptured, or otherwise made visible.}
\entry{bonua}{b\super{G}O\textsubbridge{n}\super{G}u\textsubarch{@}} \definition{adj.}{Interior, being within; inside of anything; internal; inner; further toward a center.}
\entry{bra}{b\super{G}R@\textsubarch{i}} \definition{n.}{Match, a game or contest in which two or more contestants or teams oppose each other.} \definition{num.}{Five.}
\entry{bré}{b\super{j}R\super{j}e:} \definition{n.}{Region, an extensive continuous part of a surface, space, or body.}
\entry{bria}{b\super{j}R\super{j}i\textsubarch{@}} \definition{prep.}{Except, with the exclusion of; excluding; save; but. Also refers to Besides, moreover; furthermore; also.}
\entry{bró}{b\super{G}R\super{G}o:} \definition{n.}{Result, to spring, arise, or proceed, as a consequence of actions, circumstances, premises, etc.; be the outcome.}
\entry{brug}{b\super{G}R\super{G}Ug} \definition{n.}{Case, an instance of the occurrence, existence, etc., of something.}
\entry{bú}{b\super{G}u:} \definition{n.}{Spring, the season between winter and summer: in the Northern Hemisphere from the vernal equinox to the summer solstice; in the Southern Hemisphere from the autumnal equinox to the winter solstice.}
\entry{bud}{b\super{G}U\textsubbridge{d}\super{G}} \definition{adj.}{Significant, having or expressing a meaning; indicative.}
\end{description}
\end{multicols}
\end{document}